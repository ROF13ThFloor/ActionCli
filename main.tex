\documentclass[12pt,a4paper]{article}
% \usepackage[english]{babel}
% \usepackage[utf8x]{inputenc}

\usepackage{graphicx} % Required for inserting images.
\usepackage[margin=25mm]{geometry}
\parskip 4.2pt  % Sets spacing between paragraphs.
% \renewcommand{\baselinestretch}{1.5}  % Uncomment for 1.5 spacing between lines.
\parindent 8.4pt  % Sets leading space for paragraphs.
\usepackage[font=sf]{caption} % Changes font of captions.

\usepackage{amsmath}
\usepackage{amsfonts}
\usepackage{amssymb}
\usepackage{siunitx}
\usepackage{verbatim}
\usepackage{hyperref} % Required for inserting clickable links.
\usepackage{natbib} % Required for APA-style citations.


\title{self-reflection 2 Your Next Step as a Teacher}
\author{Mojtaba Moazen}

\begin{document}
\maketitle
\section{self-reflection}
After the course has been started, I realized a lot of new concepts in the area of Teaching and Learning. Concerning the learning environment feature, my knowledge has been improved by knowing important aspects of that. Upon reflecting on my beliefs, I realized that I initially viewed teaching as a one-way street, where the instructor's role was to impart knowledge to students without much regard for their diverse learning styles and preferences. Additionally, I saw teaching as a process where the instructor wields authority and controls the information flow in the classroom and is also responsible for shaping a good natural learning environment with specific features \cite{ain2011best}. 

As the course continued, my views on teaching went through a profound change. A key turning point was when we explored sustainable development and its impacts on education. This part prompted me to deeply reflect on my role as a teacher, emphasizing not just academic development but also considering teaching as a global impact. I began to understand the close link between my knowledge in the subject and broader societal aims such as sustainability, leading me to see the importance of incorporating sustainability principles into my teaching methods \cite{wiek2011key}.


During the activities within this course, such as conducting interviews, I came to a profound realization regarding the pivotal role of feedback in shaping the learning environment during teaching. Through the interview process, it became evident that providing feedback on student activities is paramount for fostering a conducive learning atmosphere. Additionally, being open to receiving feedback on one's teaching methodology is equally essential for continuous improvement \cite{timperley2007power}.

In oursd ds  interviews, both the teacher and the Ph.D. student emphasized the significance of feedback in the learning process. This underscores the importance of this concept in creating an effective learning environment. Feedback serves as a mechanism for facilitating growth and development, both for the students and the instructor. It allows educators to assess student progress, identify areas for improvement, and tailor instruction to meet individual learning needs

Undoubtedly, this transformative journey was not without its challenges. One of the most significant hurdles I encountered was overcoming ingrained teaching habits and overcoming resistance to change. However, through self-reflection and ongoing dialogue with my group members, I gradually learned to embrace innovation and experimentation in my teaching practices. By acknowledging and addressing my limitations, I was able to cultivate a growth mindset and continuously strive for improvement in my instructional methods. I attempted to practice and overcome this challenge by teaching through a video activity that was assigned during the course. 

Moving forward, I am dedicated to utilizing the knowledge and understanding acquired from this course to improve my teaching methods. One significant aspect I will change in my teaching approach is to not only focus on transmitting knowledge to students but also to consider other elements such as sustainability education and reflective practice. With the knowledge gained from this course, I aim to create dynamic learning environments that inspire curiosity, creativity, and lifelong learning among my students. This course has been a transformative journey that has challenged and reshaped my perspectives on teaching in higher education. Through critical reflection and engagement with course objectives, I have transitioned from being a traditional educator to becoming a proactive agent of change committed to fostering inclusive, empowering, and sustainable learning communities.

\bibliographystyle{apalike}
\bibliography{example}

\end{document}
